\documentclass{beamer} 
\usepackage{amsmath,amsthm}
\usepackage{amssymb}
\usepackage[english]{babel}
\usepackage{latexsym}
\usepackage{amsfonts}
\usepackage{graphicx}
\usepackage{float}
\usepackage{graphics}
\usepackage{epsfig}
\usepackage{url}
\usepackage{soul}

\usetheme{WVU}
\usecolortheme{WVU}
\usepackage{multirow}

\mode<presentation> 

\title[VE401 SU2022 RC intro]{VE401 SU2022 RC: introduction}

\author[ Shuyu Wu ]{ Shuyu Wu }
\institute[UM-SJTU JI]{UM-SJTU Joint Institute \vspace{.2cm} \\ \includegraphics[scale=0.3]{umji_logo.png}\\wushuyu2002@sjtu.edu.cn}
\date[May 2022]{\today}

\begin{document}
\begin{frame} 

\titlepage 

\end{frame} 

\begin{frame}
       \frametitle{Outline}
       \tableofcontents
\end{frame}

\section{Introduction} 
\begin{frame} 
\frametitle{Introduction to ...} 

\begin{itemize}
\item Instructor
\item<2-> Me
\item<3-> Course Content
\end{itemize} 
\end{frame} 

\begin{frame}
    \frametitle{Instructor}
    Horst Hohberger
    \begin{itemize}
        \item Teaching Professor at JI(2021-)
        \item Join JI since 2007
        \item Personal Website: https://umji.sjtu.edu.cn/\textasciitilde horst/index.html
            \begin{itemize}
                \item The course slide and sample assignments are available here(but isn't the assignment this year)
            \end{itemize}
    \end{itemize}
\end{frame}

\begin{frame}
    \frametitle{Self Introduction}
    Shuyu Wu
    \begin{itemize}
        \item work experience at JI
            \begin{itemize}
                \item TA in VG101 FA21(2021.9-2021.12)
                \item Honor Council Member(2021.11-)
            \end{itemize}
        \item the programming language I like for this course
            \begin{itemize}
                \item Matlab
                \item Python(with scipy)
                \item \st{Mathematica}(but it's the official language for this course)
            \end{itemize}
    \end{itemize}
\end{frame}

\begin{frame}
    \frametitle{About this course, VE401}
    Three main parts
    \begin{itemize}
        \item probability theory
        \item introduction to statistics
        \item analysis of linear regression and analysis of variance
    \end{itemize}
    We don't teach
    \begin{itemize}
        \item random process
        \item Bayesian analysis in statistical inference
    \end{itemize}
    But there may be extra topics in my RC related to these topics

\end{frame}

\begin{frame}
    \frametitle{About this course, VE401}
    
    \begin{itemize}
        \item What's ``probability" and ``statistics" based on the probability and statistics knowledge you have learnt in your life?
        \item Is there a mathematical definition for ``probability"?
        \item What's the purpose of statistics? How can it help us understand and transform the world?
        \item Is statistics theory belongs to mathematics? Why?
    \end{itemize}
\end{frame}

\begin{frame}
    \frametitle{Example}
    \begin{itemize}
        \item In physics lab, we calcuate type-A uncertainty by calcuating the sample deviation $S$, and use formula $t_{n-1}\frac{S}{\sqrt{n}}$. Have you ever thought what's the principle behind the formula?(\st{I know you haven't. This course is so annoying.} But now you have survived this course and let's revisit these contents again).
        \item In COVID-19 Hong Kong epidemic(2022 Feb-April), research has shown that the death rate for people who don't get vaccined is 2.87\%, while 0.14\% for fully vaccined(2 doses) and 0.03\% for boosted. Does that really show vaccine is so powerful?
        \item In USA, before general election, there'll be polls. However, for many times we can find that the result is different from polls. What factors may contribute to the difference?
    \end{itemize}
    

\end{frame}

\section{My RC Plan}
\begin{frame}
    \frametitle{RC Plan}
    My recitation class will contain these parts:
    \begin{itemize}
        \item review of concepts
        \item exercise
        \item subjective view on some ideas
        \item extra topics and experiments(computer simulation)
    \end{itemize}
\end{frame}

\begin{frame}
    \frametitle{Extra Topics}
    Extra topics may include, but not limited to,
    \begin{itemize}
        \item history of some famous mathematicians or concepts
        \item introduction to random process
        \item game theory and gambling
        \item some python package
        \item related algorithms
    \end{itemize}
    
\end{frame}

\section{Concept Checking Paper}
\begin{frame}
    \frametitle{Concept Checking Paper}
    Weekly concept revision. Initiated by previous VV186, 285, 286TA Leyang Zhang. Also used in VE203 SP2022.\par
    \vspace{0.3cm}
    Every week I'll post next week's concept checking paper. I may not release all the materials altogether at the beginning of the semester.



\end{frame}


\section{More resources}

\begin{frame} 

\frametitle{About Programming in this Course}
Official language: Mathematica
\begin{itemize}
    \item powerful in mathematic calculation(like integral)
    \item it's not usual ``program", it's notebook
    \item may be in a mess if you don't use it properly
    \item alternative: python, matlab, R, SPSS(if you happen to know how to use it)
\end{itemize}
Python
\begin{itemize}
    \item open source libraries
    \item for this course: numpy, scipy, pandas
    \item jupyter notebook
\end{itemize}
Matlab
\begin{itemize}
    \item commercial software which you're familar with
    \item very powerful in mathematic modeling, hypothesis testing and linear regresson analysis
\end{itemize}

\end{frame}

\section{Other Comments}
\begin{frame}
    \frametitle{Other comments for this course}
    \begin{itemize}
        \item Actually much ``easier'' than VV186, 285, 286 in terms of the abstraction of concepts.
        \item Warning! If you think this course only consists of applying formula and using softwares, it will come at a cost of your grade. You need to understand the principle behind the concepts.
        \item Contain many group works. Don't violate HC otherwise the whole team may be reported to honor council together. 
        \item Also, any grade issues need to be raised within 2 weeks, otherwise they'll be automatically rejected. ``Legitimate concerns regarding grade-relevant scores communicated in a timely fashion, will of course always be treated in an objective manner. On the other hand, `hunting for points' in old assignments just so that a few tenths of a percentage point can lead to a jump into the next higher letter grade scale is not acceptable. ''
    \end{itemize}
    
\end{frame}














\end{document} 