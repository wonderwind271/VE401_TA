\documentclass{beamer} 
\usepackage{amsmath,amsthm}
\usepackage{mathrsfs}
\usepackage{amssymb}
\usepackage[english]{babel}
\usepackage{latexsym}
\usepackage{amsfonts}
\usepackage{graphicx}
\usepackage{float}
\usepackage{graphics}
\usepackage{epsfig}
\usepackage{url}
\usepackage{soul}
\usepackage{listings}
\usepackage{bm}
\usepackage{braket}

% \usepackage{minted}


\usetheme{WVU}
\usecolortheme{WVU}
\usepackage{multirow}

\mode<presentation> 

\title[VE401 SU2022 final RC(part 2)]{VE401 SU2022 final RC(part 2)}

\author[ Shuyu Wu ]{ Shuyu Wu }
\institute[UM-SJTU JI]{UM-SJTU Joint Institute \vspace{.2cm} \\ \includegraphics[scale=0.3]{umji_logo.png}\\wushuyu2002@sjtu.edu.cn}
\date[July 2022]{\today}

\begin{document}
\begin{frame} 

\titlepage 

\end{frame} 
\section{Non-Parametric Test and Paired Test}
\begin{frame}
    \frametitle{Outline}
    \tableofcontents[currentsection]
\end{frame}

\begin{frame}
    \frametitle{Non-Parametric Test}
    We use non-parametric test when:
    \begin{itemize}
        \item We don't know where the sample come from. For example, we don't know if they come from normal distribution
        \item We're not sure if the sample are independent to each other
    \end{itemize}
    There're 3 (or 4) types of non-parametric test:
    \begin{itemize}
        \item sign test
        \item Wilcoxon Signed Rank Test
        \item Wilcoxon Rank Sum Test
        \item non-parametric paired test (Wilcoxon Signed Rank Test)
    \end{itemize}

\end{frame}

\begin{frame}
    \frametitle{Sign Test for Median}

    The null hypothesis is of these kinds: 
    \begin{itemize}
        \item $H_0: M=M_0$
        \item $H_0: M\leq M_0$
        \item $H_0: M\geq M_0$
    \end{itemize}
    $M$ is the median. Now given a sample $X_1, X_2,\dots , X_n$, we want to figure out whether $H_0$ should be rejected.\par
    Set $Q_{+}$ be the number of the sample that is greater than $M$, $Q_{-}$ be the number of the sample that is smaller than $M$. We have 
    \[P[Q_{-}\leq k]=P[Q_{+}\leq k]=\frac{1}{2^n}\sum\limits_{x=0}^{k} \binom{n}{x}\]

\end{frame}

\begin{frame}
    \frametitle{P-value}

    For null hypothesis
    \begin{itemize}
        \item $H_0: M\leq M_0$ The P value is $P[Q_{-}\leq k|M=M_0]$
        \item $H_0: M\geq M_0$ The P value is $P[Q_{+}\leq k|M=M_0]$
        \item $H_0: M= M_0$ The P value is $2*P[min\{Q_{+},Q_{-}\}\leq k|M=M_0]$
    \end{itemize}
    k is the corresponding count in the sample.\par
    When the P-value is below the significance level $\alpha$, $H_0$ is rejected.

\end{frame}

\begin{frame}
    \frametitle{Wilcoxon Signed Rank Test}

    If the sample is approximately symmetric, the Wilcoxon sign rank test is more accurate.\par

    The $H_0$ is basically the same. If it's a two-tailed test, the matlab function\par
    signrank(A)\par
    will compare the sample A with 0 and return the p-value for $H_0: M=0$. To test $H_0: M=M_0$, you can have\par
    signrank($A-M_0$)\par
    You need to divide it by 2 if it's a one-tailed test.\par


\end{frame}

\begin{frame}
    \frametitle{Wilcoxon Rank Sum Test}

    Compare if two distributions have the same median, or one of them are larger.\par
    If you want to do a two-tailed test for sample A and B, use\par
    ranksum(A,B)\par
    You need to divide it by 2 if it's a one-tailed test.

\end{frame}

\begin{frame}
    \frametitle{Paired test}

    To compare the mean of $A$ and $B$, we can perform some test on $A-B$.\par
    If we perform a Wilcoxon sign rank (not rank sum) test, this is called non-parametric paired test.\par
    matlab: signrank(A,B). Again, if it's one-tailed test, divide it by 2.\par
    You can also perform T-test directly. 

\end{frame}

\begin{frame}
    \frametitle{When do We Use Paired Test}

    When $A$ and $B$ are positively correlated. Otherwise a pool test is better because it gives more degrees of freedom.

\end{frame}

\begin{frame}
    \frametitle{Correlation}

    The estimator of correlation:
    \[R:=\hat{\rho}=\frac{\sum(X_i-\overline{X})(Y_i-\overline{Y})}{\sqrt{\sum(X_i-\overline{X})^2}\sqrt{\sum(Y_i-\overline{Y})^2}}\]

    The confidence interval is 
    \[\tanh(\tanh^{-1}(R)\pm \frac{z_{\alpha/2}}{\sqrt{n-3}})\]
    The critical region is just the complement set.\par
    Or, use the test statistic
    \[Z=\sqrt{n-3}(\tanh^{-1}(R)-\tanh^{-1}(\rho_0))\]

\end{frame}

\section{Categorical Data}
\begin{frame}
    \frametitle{Outline}
    \tableofcontents[currentsection]
\end{frame}

\begin{frame}
    \frametitle{Chi-squared Goodness-of-fit Test}

    \textbf{The Pearson Statistic}: Let $O_i$ be the observed value and $E_i$ be the expected value. Pearson statistic is 
    \[\sum\limits_{i=1}^{k}\frac{(O_i-E_i)^2}{E_i}\]
    $O_i$ must be original value.

\end{frame}

\begin{frame}
    \frametitle{Goodness-of-Fit Test for a Discrete Distribution}
    Let $m$ be the number of parameters to be estimated. \par
    The Pearson statistic follows a chi-squared distribution with $\chi^2_{k-1-m}$ degrees of freedom.\par
    For continuous distribution you need to split them into bins.

\end{frame}

\begin{frame}
    \frametitle{Test of Independence}

    If two groups of data are independent, their joint distribution is the product of marginal density. So, we can calculate out the marginal density by summing one row/column up and divide by total sample size.\par
    Suppose two categories are independent, then the joint density should be their product, and we can calculate the ``expected value''. Finally we can perform the chi-squared test. The degree of freedom is $(r-1)(c-1)$.

\end{frame}

\section{Simple Linear Regression}
\begin{frame}
    \frametitle{Outline}
    \tableofcontents[currentsection]
\end{frame}
\begin{frame}
    \frametitle{Estimate of Parameters}

    Encouraged tool: matlab (with curve fitting toolbox)\par
    Set 
    \[\overline{x}=\frac{1}{n} \sum\limits_{i=1}^{n} x_i, \overline{y}=\frac{1}{n} \sum\limits_{i=1}^{n} y_i\]
    \[S_{xx}=\sum\limits_{i=1}^{n}(x_i-\overline{x})^2, S_{yy}=\sum\limits_{i=1}^{n}(y_i-\overline{y})^2, S_{xy}=\sum\limits_{i=1}^{n}(x_i-\overline{x})(y_i-\overline{y})\]
    And we have 
    \[b_1=\frac{S_{xy}}{S_{xx}}, b_0=\overline{y}-b_1 \overline{x}, SS_E=S_{yy}-b_1 S_{xy}\]

\end{frame}

\begin{frame}
    \frametitle{Least Squares Estimator for the Variance}

    \[S^2=\frac{SS_E}{n-2}\]
    The confidence interval for $\beta_0$ and $\beta_1$ are:
    \[\beta_1: b_1\pm t_{\alpha/2, n-2} \frac{S}{\sqrt{S_{xx}}}\]
    \[\beta_0: b_0\pm t_{\alpha/2, n-2} \frac{S \sqrt{\sum x_i^2}}{\sqrt{n S_{xx}}}\]
    We say that a regression is significant if there is statistical evidence that
    the slope $\beta_1\neq 0$, that is to say, the confidence interval of $\beta_1$ doesn't contain 0.

\end{frame}

\begin{frame}
    \frametitle{Confidence Interval and Prediction Interval}

    Confidence Interval the Estimated Mean $\mu_{Y|x}$: 
    \[\hat{\mu}_{Y|x}\pm t_{\alpha/2, n-2}S\sqrt{\frac{1}{n}+\frac{(x-\overline{x})}{S_{xx}}}\]
    Please notice that $\mu_{Y|x}$ is a parameter, not a random variable.\par
    Now we want to know what we will get for $Y|x$ next time we get X=x, which is a random variable. That is to say, we want the prediction of the random variable.\par
    The $100(1-\alpha)\%$ prediction interval for $Y|x$ is given by
    \[\widehat{Y|x}\pm t_{\alpha/2, n-2}S\sqrt{1+\frac{1}{n}+\frac{(x-\overline{x})}{S_{xx}}}\]
    The $\widehat{Y|x}=\hat{\mu}_{Y|x}=b_0+b_1 x$

\end{frame}

\begin{frame}
    \frametitle{Coefficient of Determination}

    $SS_{T}:=S_{yy}$ is the Total Sum of Squares. $SS_{E}:=\sum\limits_{i=1}^{n} e_i^2=\sum\limits_{i=1}^{n} (y_i-(b_0+b_1 x))^2$ represents the variation of Y that remains after we have applied the
    model.\par
    \[R^2:=1-\frac{SS_{E}}{SS_{T}}\]
    Also we have 
    \[R^2=\frac{S_{xy}^2}{S_{xx}S_{yy}}\]
    

\end{frame}

\begin{frame}
    \frametitle{T and F-test for significance of regression}
    We can rewrite the significance of regression:
    \[T_{n-2}=\frac{R}{\sqrt{1-R^2}}\sqrt{n-2}\]
    Reject at $|T_{n-2}|>t_{\alpha/2,n-2}$\par
    Or we can use F-test (which can be generalized to multiple linear regression).\par
    \[F_{1,n-2}=(n-2)\frac{R^2}{1-R^2}=(n-2)\frac{SS_{T}-SS_{E}}{SS_{E}}\]
    Reject $H_0: \beta_1=0$ at significance level $\alpha$ when $F_{1,n-2}>f_{\alpha,1, n-2}$. Note that this is a one-tailed test.\par
    

\end{frame}

\begin{frame}
    \frametitle{Lack-of-Fit and Pure Error}

    \[F_{k-2,n-k}=\frac{SS_{E;lf}/(k-2)}{SS_{E;pe}/(n-k)}\]
    And reject $H_0: \text{the model is proper}$ at significance level $\alpha$ when $F_{k-2,n-k}>f_{\alpha,k-2,n-k}$.\par
    Sum up all the internal sum of square, we get the error sum of squares due to pure error, $SS_{E;pe}$.\par
    \[SS_{E;pe}:=\sum\limits_{i=1}^{k}\sum\limits_{j=1}^{n_i}(Y_{ij}-\overline{Y_i})^2\]
    And $\frac{SS_{E;pe}}{\sigma^2}$ follows $\chi^2_{n-k}$. Also $SS_{E;lf}$ (the error sum of squares due to lack of fit) is defined as
    \[SS_{E;lf}:=SS_E-SS_{E;pe}\]

\end{frame}

\section{Multiple Linear Regression}
\begin{frame}
    \frametitle{Outline}
    \tableofcontents[currentsection]
\end{frame}
\begin{frame}
    \frametitle{Model Specification Matrix}
    There's basically no difference between different multiple linear models.\par
    For the model $Y_i=\beta_0+\beta_1 x_i+\dots +\beta_p x_i^p + E_i$, $i=1,2,\dots , n$, we can use a matrix equation to represent
    \[\mathbf{Y}=X\boldsymbol{\beta}+\mathbf{E}\]
    And 
    \begin{equation*}      
        \mathbf{Y}=
        \left( 
          \begin{array}{ccc}  
        
            Y_1\\
            \vdots\\
            Y_n
          \end{array}
        \right)   
        , X=
        \left(
        \begin{array}{ccccc}
            1 & x_1 & x_1^2 & \cdots & x_1^p\\
            \vdots & \vdots & \vdots & \ddots & \vdots\\
            1 & x_n & x_n^2 & \cdots & x_n^p
        \end{array}
        \right)
        , \mathbf{E}=
        \left(
            \begin{array}{ccc}  
        
                E_1\\
                \vdots\\
                E_n
              \end{array}
        \right)
    \end{equation*}
    The $X$ is called the model specification matrix. Each row is one sample, and each column corresponds to one term in the model.\par
    Make sure you also know how to calculate other model's specification matrix.

\end{frame}

\begin{frame}
    \frametitle{The Regression Coefficients}

    Now we want to calculate $\mathbf{b}$ which is the estimator of $\boldsymbol{\beta}$. Our result is
    \[\mathbf{b}=(X^{T}X)^{-1}X^{T}\mathbf{Y}\]
    Please notice that this applies to any linear regressions, not only for polynomial model.\par
    Of course, you'd better do the calculation with matlab or other softwares.

\end{frame}

\begin{frame}
    \frametitle{Error Analysis, $R^2$}

    The P projection is an $n*n$ matrix with value $\frac{1}{n}$ for each of the element. The hat matrix $H:=X(X^{T}X)^{-1}X^{T}$. Our result for $SS_{T}$ and $SS_{E}$ are:
    \[SS_{E}=\Braket{Y, (\mathbb{I}_{n}-H)Y}, \]
    \[SS_{T}=\Braket{Y,(\mathbb{I}_{n}-P)Y}=SS_{E}+SS_{R}\]
    \[R^2=1-\frac{SS_{E}}{SS_{T}}\]
    \textbf{Notice:} You shouldn't use mathematica to calculate $R^2$ in function "NonLinearModelFit". Mathematica will use corrected $R^2$ which is not taught in our course. However, matlab will give correct answer.

\end{frame}

\begin{frame}
    \frametitle{Distribution of the Sum of Squares Error}

    \[\frac{SS_E}{\sigma^2}\sim \chi^2_{n-p-1}\]
    $p$ is the number of variables in the regression model($x$ and $x^2$ term is counted as 2). $p+1$ is the number of terms. For example, for simple linear regression $p=1$ $(p+1=2)$, the model specification matrix has n rows and $(p+1)$ columns.\par
    Also, the estimator 
    \[S^2=\frac{SS_E}{n-p-1}\]
    is unbiased for $\sigma^2$

\end{frame}

\begin{frame}
    \frametitle{F-Test for Significance of Regression}
    The null hypothesis is $H_0: \beta_1=\beta_2=\dots  =\beta_p=0$\par
    The test statistic for the F-test is 
    \[F_{p,n-p-1}=\frac{SS_R/p}{S^2}=\frac{n-p-1}{p}\frac{R^2}{1-R^2}\]
    So you can easily determine the F statistic using only matlab. And it's a one-tailed test of course, so we reject $H_0$ at $F_{p,n-p-1}>f_{\alpha, p,n-p-1}$.\par


\end{frame}

\begin{frame}
    \frametitle{Analysis of the Parameters}

    $Var[\textbf{b}]=\sigma^2(X^{T}X)^{-1}$. Denote $\xi_{ii}=((X^{T}X)^{-1})_{ii}$, we have 
    \[Var[B_i]=\xi_{ii}\sigma^2\]
    In practice, we use $S^2$ to estimate $\sigma^2$, so it'll be a T-test. The confidence interval for the parameter $\beta_j$ is
    \[b_j\pm t_{\alpha/2, n-p-1}S\sqrt{\xi_{jj}}\]
    for $j=0$ to $p$.

\end{frame}

\begin{frame}
    \frametitle{Confidence Intervals for the Estimated Mean}

    Short review question: What's the difference between confidence interval and prediction interval?\par
    \vspace{0.3cm}

    A $100(1-\alpha)\%$ confidence interval for $\mu_{Y|x_0}$ is 
    \[\mu_{Y|x_0}=\hat{\mu}_{Y|x_0}\pm t_{\alpha/2, n-p-1} S \sqrt{\mathbf{x_0}^{T}(X^{T}X)^{-1}\mathbf{x_0}}\]
    Here, the $\mathbf{x_0}\in \mathbb{R}^{p+1}$\par

    A $100(1-\alpha)\%$ prediction interval for $Y|x_0$ is 
    \[Y|x_0=\widehat{Y|x_0}\pm t_{\alpha/2, n-p-1} S \sqrt{1+\mathbf{x_0}^{T}(X^{T}X)^{-1}\mathbf{x_0}}\]
    Note that $\hat{\mu}_{Y|x_0}=\widehat{Y|x_0}$ is just $X\mathbf{b}$

\end{frame}

\begin{frame}
\frametitle{Partial F-Test for Model Sufficiency}

    Some times a reduced model is good enough compared with full model. That is to say, $SS_{E,reduced}$ is not much bigger than $SS_{E, full}$.\par
    We set $H_0: \text{the reduced model is sufficient}$. We reject $H_0$ if there's evidence that $SS_{E,full}\ll SS_{E,reduced}$.\par
    \vspace{0.3cm}
    Now suppose that we have a full model of $p+1$ predictor variables, and a reduced model with $m+1$ ($m<p$) predictor variables.\par
    \vspace{0.3cm}
    The test statistic is $F_{p-m,n-p-1}=\frac{n-p-1}{p-m}\frac{SS_{E,reduced}-SS_{E,full}}{SS_{E,full}}$. Again this must be a one-tailed test, so we reject $H_0$ at significance level $\alpha$ if $F_{p-m,n-p-1}>f_{\alpha,p-m,n-p-1}$.\par
    \vspace{0.3cm}
    Partial F-test is actually the generalization of both T-test(for the significance of a single parameter) and F-test (for the significance of the whole model).

\end{frame}

\section{Q\&A}
\begin{frame}
    \frametitle{Outline}
    \tableofcontents[currentsection]
\end{frame}
\begin{frame}
    \frametitle{Some Exercise in Sample Exam and Last Year's Final}

    

\end{frame}

\begin{frame}
    \frametitle{Python And Matlab Cheatsheet}

    Uploaded on canvas

\end{frame}

\begin{frame}
    \frametitle{Farewell}

    There is one in a billion course you take in JI, I hope this is the one that help you understand and transform the world. Also, one in a trillion TA you meet in JI, we hope we're the TA that actually help you.\par
    \vspace{0.3cm}
    ``I love math, love this job, and love you all!"\footnote{Chengsong Zhang. VV285 SU2021 Final RC}\par
    
    

\end{frame}
\begin{frame}
    \frametitle{Q\&A}
    Good luck to all of you for your final. Maybe see you in EECS 501(probability and random process) next winter semester.\par
    \vspace{0.3cm}
    Also, hope you all enjoy this course (though it is okay if you don't)\par
    Feel free to ask if you have any questions.\par
    
    
    
\end{frame}

\end{document} 